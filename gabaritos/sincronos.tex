\section*{Sistemas Síncronos e Elementos de Memória}
\setcounter{exercicio}{0}

\gabaritoex{1}
Circuitos síncronos utilizam um sinal de clock para sincronizar mudanças de
estado, enquanto circuitos assíncronos respondem imediatamente às variações
de entrada.

\gabaritoex{2}
O clock define o ritmo de operação do sistema e garante sincronização entre os
elementos sequenciais.

\gabaritoex{3}
Latches são sensíveis ao nível do sinal de controle, enquanto flip-flops mudam
de estado apenas na borda do clock.

\gabaritoex{4}
O flip-flop tipo D armazena o valor da entrada D na borda ativa do clock.

\gabaritoex{5}
A tabela verdade do FF-D mostra que a saída assume o valor de D apenas na
transição do clock.

\gabaritoex{6}
Q(A) = 0,\quad
Q(B) = 1,\quad
Q(C) = 0,\quad
Q(D) = 0,\quad
Q(E) = 1


\gabaritoex{7}
Um registrador é um conjunto de bits capaz de armazenar um valor binário.

\gabaritoex{8}
O sinal de endereço indica qual posição da memória será acessada.

\gabaritoex{9}

\begin{center}
\begin{tabular}{|c|c|c|}
\hline
\textbf{Clock} & \textbf{Entrada} & \textbf{Bit armazenado} \\ \hline
$\uparrow$ & 0 & 0 \\ 
$\uparrow$ & 1 & 1 \\ 
0/1        & X & mantém valor \\ \hline
\end{tabular}
\end{center}

\textit{O Binary Digit armazena o valor da entrada apenas na borda de subida
do clock, mantendo o valor anterior nos demais instantes.}


\newpage