\section*{Álgebra Booleana}
\setcounter{exercicio}{0}

\gabaritoex{1}
Portas lógicas correspondentes: AND, NAND, OR, NOR, NOT, XOR e XNOT.

\gabaritoex{2}
(a) A tabela representa a porta XOR, pois a saída é 1 quando exatamente uma das
entradas é 1.  
(b) A tabela representa a porta NAND, pois apenas quando ambas as entradas são 1
a saída é 0.

\gabaritoex{3}
Aplicando a distributividade:
\[
A(B+\overline{C}+D)=AB+A\overline{C}+AD
\]

\gabaritoex{4}
Pelo teorema de De Morgan:
\[
\overline{A+B+C}=\overline{A}\,\overline{B}\,\overline{C}
\]

\gabaritoex{5}
As simplificações incorretas são aquelas que violam a distributividade ou a
absorção, em especial quando fatores são omitidos durante a expansão. Nesse caso, as alternativas 1 e 3

\gabaritoex{6}
A propriedade falsa é a alternativa que não pode ser deduzida a partir das leis
fundamentais da álgebra booleana. A alternativa 2

\gabaritoex{7}
A forma canônica correta é:
\[
Q=\overline{A}\,\overline{B}+AB
\]

\gabaritoex{8}
A expressão simplificada corresponde a:
\space

PoS:
\[
Q = (A + B + \overline{C})
    (A + \overline{B} + C)
    (\overline{A} + B + \overline{C})
    (\overline{A} + \overline{B} + C)
\]

SoP:
\[
Q = \overline{A}\,\overline{B}\,\overline{C}
  + \overline{A} B C
  + A \overline{B}\,\overline{C}
  + A B C
\]
\[Q = 
BC((\overline{A} + A) + \overline{BC}(\overline{A} + A)\]
\[Q = BC + \overline{BC}\]

\begin{figure}
    \centering
    \includegraphics[width=0.5\linewidth]{fotos/circuito1.png}
    \label{fig:placeholder}
\end{figure}

\gabaritoex{9}
4 combinações já que duas são iguais

\gabaritoex{10}
A principal característica da forma canônica SoP é permitir a implementação do
circuito utilizando apenas portas AND, OR e inversores.

\gabaritoex{11}
\[1.\overline{BC} + B\overline{C} + \overline{A}BC\]
\[2.\overline{C}(\overline{B}+B) + \overline{A}BC\]
\[3.\overline{C} + \overline{A}BC\]

\gabaritoex{12}
Mapa 1: \[B+A\]
Mapa 2: \[A\overline{B} + \overline{B}C + \overline{A}B\]
Mapa 3: \[\overline{BCD} +A\overline{BC} + BC\overline{D}\]
Mapa 4: \[A + C\overline{D} + \overline{BD}\]

\gabaritoex{13}

\begin{figure}[H]
    \centering
    \includegraphics[width=0.5\linewidth]{fotos/mapa2.png}
\end{figure}

\[
Q = \overline{A} + \overline{B} + \overline{C} + \overline{D}
\]


\gabaritoex{14}
\[
\begin{array}{c c c|c}
A & B & C & Q \\ \hline
0 & 0 & 0 & 0 \\
0 & 0 & 1 & 0 \\
0 & 1 & 0 & 0 \\
0 & 1 & 1 & 1 \\
1 & 0 & 0 & 0 \\
1 & 0 & 1 & 1 \\
1 & 1 & 0 & 0 \\
1 & 1 & 1 & 1
\end{array}
\]

\[
Q = \overline{A}BC + A\overline{B}C + ABC
\]

\[
Q = BC(\overline{A} + A) + A\overline{B}C
\]
\[
Q = BC + A\overline{B}C
\]


\newpage