\section*{Dados Digitais}
\setcounter{exercicio}{0}

\gabaritoex{1}
O endereço \texttt{0x0C} \[(12)\] contém o valor binário:
\[
0b11111011
\]
que corresponde ao valor decimal 251.

\gabaritoex{2}
Os bytes capturados são:
\[
0x5A,\;0x69,\;0x67,\;0x61
\]
que correspondem aos caracteres ASCII \texttt{Ziga}.

\gabaritoex{3}
Os valores exibidos no display de sete segmentos são:
\[
1,\;0,\;2
\]

\gabaritoex{4}
\begin{itemize}
  \item nibble: 4 bits
  \item byte: 8 bits
  \item halfword: 16 bits
  \item word: 32 bits
\end{itemize}

\gabaritoex{5}
Conversões corretas para binário, com número mínimo de bits, conforme tabela
padrão de potências de dois

\begin{center}
\begin{tabular}{|c|c|c|}
\hline
\textbf{Decimal} & \textbf{nº de bits} & \textbf{Binário} \\ \hline
0     & 1  & $0b0$ \\
1     & 1  & $0b1$ \\
2     & 2  & $0b10$ \\
3     & 2  & $0b11$ \\
4     & 3  & $0b100$ \\
5     & 3  & $0b101$ \\
6     & 3  & $0b110$ \\
7     & 3  & $0b111$ \\
8     & 4  & $0b1000$ \\
9     & 4  & $0b1001$ \\
10    & 4  & $0b1010$ \\
115   & 7  & $0b1110011$ \\
256   & 9  & $0b100000000$ \\
1027  & 11 & $0b10000000011$ \\ \hline
\end{tabular}
\end{center}
.

\gabaritoex{6}
\[
0b0=0,\quad 0b100=4,\quad 0b10011=19,\quad 0b11111=31,\quad 0b01010=10
\]

\gabaritoex{7}
\begin{center}
\begin{tabular}{|c|c|}
\hline
\textbf{Decimal} & \textbf{Hexadecimal} \\ \hline
0  & $0x0$  \\
1  & $0x1$  \\
2  & $0x2$  \\
3  & $0x3$  \\
4  & $0x4$  \\
5  & $0x5$  \\
6  & $0x06$ \\
7  & $0x7$  \\
8  & $0x8$  \\
9  & $0x9$  \\
10 & $0xA$  \\
11 & $0xB$  \\
12 & $0xC$  \\
13 & $0xD$  \\
14 & $0xE$  \\
15 & $0xF$  \\
16 & $0x10$ \\
17 & $0x11$ \\
18 & $0x12$ \\ \hline
\end{tabular}
\end{center}


\gabaritoex{8}
\[
0b11110=0x1E,\quad 0b101=0x5,\quad 0b10100011=0xA3,\quad 0b11010=0x1A, \quad 0b00000=0x0
\]

\gabaritoex{9}
\[
0x0003=3,\quad 0xA=10,\quad 0x55=85,\quad 0x0101=257
\]

\gabaritoex{10}

\begin{center}
\begin{tabular}{|c|c|}
\hline
\textbf{Decimal} & \textbf{Octal} \\ \hline
0  & $0o0$  \\
1  & $0o1$  \\
2  & $0o2$  \\
3  & $0o3$  \\
4  & $0o4$  \\
5  & $0o5$  \\
6  & $0o06$ \\
7  & $0o7$  \\
8  & $0o10$ \\
9  & $0o11$ \\
10 & $0o12$ \\
11 & $0o13$ \\
12 & $0o14$ \\
13 & $0o15$ \\
14 & $0o16$ \\
15 & $0o17$ \\
16 & $0o20$ \\
17 & $0o21$ \\
18 & $0o22$ \\ \hline
\end{tabular}
\end{center}


\gabaritoex{11}

\begin{center}
\begin{tabular}{|c|c|}
\hline
\textbf{Decimal} & \textbf{BCD} \\ \hline
50  & $0101\ 0000$ \\
1   & $0001$ \\
103 & $0001\ 0000\ 0011$ \\
904 & $1001\ 0000\ 0100$ \\
4   & $0100$ \\
5   & $0101$ \\ \hline
\end{tabular}
\end{center}


\gabaritoex{12}
\begin{center}
\begin{tabular}{c|c}
\hline
\textbf{Binário} & \textbf{Decimal} \\
\hline
0b100 & 4 \\
0b0 & 0 \\
0b10011 & 19 \\
0b11111 & 31 \\
0b01010 & 10
\end{tabular}
\end{center}

\[
`1'=00110001,\;`a'=01100001,\;`A'=01000001,\;`Z'=01011010,\;`0'=00110000
\]
\newpage