\section{Linguagem de Máquina}
\setcounter{exercicio}{0}


\subsection{Conjunto de Instruções}

\exercicio{Classificação de comandos quanto ao suporte do hardware}

Classifique os comandos a seguir como \textbf{Suportado} ou
\textbf{Não suportado} pelo hardware Z01.1:

\begin{itemize}
  \item \texttt{leaw \$423, \%A}
  \item \texttt{leaw \$1, \%D}
  \item \texttt{movw \$-1, (\%A)}
  \item \texttt{leaw \$-15, \%A}
  \item \texttt{movw (\%A), \%A, \%D}
  \item \texttt{incw (\%A)}
  \item \texttt{addw (\%A), \%D, (\%A)}
  \item \texttt{jle \%A}
  \item \texttt{movw \$2, \%D}
  \item \texttt{movw \$1, \%D, \%A}
  \item \texttt{addw (\%A), \%D, \%D, \%A}
  \item \texttt{movw \%D, \%S}
\end{itemize}

\espacoresposta{3cm}

% --------------------------------------------------

\exercicio{Classificação de comandos em A ou C}

Classifique os comandos a seguir como instrução do tipo A ou do tipo C:

\begin{itemize}
  \item \texttt{leaw \$1, \%A}
  \item \texttt{movw \%D, \%A}
  \item \texttt{incw \%A}
  \item \texttt{jmp}
\end{itemize}

\newpage

\exercicio{Tradução da instrução \texttt{leaw}}

A instrução:
\[
\texttt{leaw \$5, \%A}
\]
é traduzida para qual palavra em binário?

\begin{itemize}
  \item \texttt{000000000000000000}
  \item \texttt{100000000000000101}
  \item \texttt{000000000000000101}
  \item \texttt{010101010101010101}
\end{itemize}

\espacoresposta{3cm}

% --------------------------------------------------

\exercicio{Tradução da instrução \texttt{movw}}

A instrução:
\[
\texttt{movw \%D, \%A}
\]
é traduzida para qual palavra em binário?

\begin{itemize}
  \item \texttt{1 000 1 001100 0 001 000}
  \item \texttt{1 000 0 001100 0 001 000}
  \item \texttt{0 000 1 001100 0 001 000}
  \item \texttt{1 000 0 000000 0 000 011}
\end{itemize}

\espacoresposta{3cm}

% --------------------------------------------------

\exercicio{Tradução da instrução \texttt{andw}}

A instrução:
\[
\texttt{andw \%A, \%D, \%D}
\]
é traduzida para qual palavra em binário?

\begin{itemize}
  \item \texttt{1 000 0 000000 0 010 000}
  \item \texttt{1 000 1 000000 0 010 000}
  \item \texttt{1 000 0 000000 0 001 000}
  \item \texttt{0 000 0 000000 0 010 001}
\end{itemize}

\espacoresposta{3cm}

% --------------------------------------------------

\exercicio{Tradução da instrução \texttt{andw} com dois destinos}

A instrução:
\[
\texttt{andw \%A, \%D, \%D, \%A}
\]
é traduzida para qual palavra em binário?

\begin{itemize}
  \item \texttt{1 000 0 000000 0 010 000}
  \item \texttt{1 000 0 000000 0 011 000}
  \item \texttt{1 000 0 000000 0 001 000}
  \item \texttt{0 000 0 000000 0 010 001}
\end{itemize}

\espacoresposta{3cm}

% --------------------------------------------------

\exercicio{Tradução da instrução \texttt{jmp}}

A instrução \texttt{jmp} é traduzida para qual palavra em binário?

\begin{itemize}
  \item \texttt{1 000 0 XXXXXX 0 000 111}
  \item \texttt{1 000 0 XXXXXX 0 001 000}
  \item \texttt{0 000 0 XXXXXX 0 000 111}
  \item \texttt{1 000 0 XXXXXX 0 000 000}
\end{itemize}

\espacoresposta{3cm}

% --------------------------------------------------

\exercicio{Interpretação de código em linguagem de máquina}

O código a seguir está em linguagem de máquina do processador Z01. Considere que, neste hardware:
\begin{itemize}
  \item o endereço \texttt{RAM[16384]} corresponde às chaves de entrada (\texttt{SW});
  \item o endereço \texttt{RAM[16385]} corresponde aos LEDs de saída.
\end{itemize}

O que o código abaixo faz?

\begin{center}
\ttfamily
\begin{tabular}{l}
000101001011000001 \\
100011100000010000 \\
000101001011000000 \\
100000011000100000 \\
000000000000000000 \\
100000011000000111 \\
100001010100000000
\end{tabular}
\end{center}

\begin{enumerate}
  \item Move o valor da \texttt{RAM[3]} para \texttt{RAM[2]} deixando-o negativo
  \item Escreve um pixel no LCD
  \item Copia o valor das chaves \texttt{SW} para os LEDs
  \item Adiciona o valor de \texttt{RAM[1]} com \texttt{RAM[2]}
\end{enumerate}

\textit{Comentário:} este exercício explora o conceito de periféricos
mapeados em memória, em que dispositivos de entrada e saída compartilham
o mesmo espaço de endereçamento da RAM.


\espacoresposta{2cm}

