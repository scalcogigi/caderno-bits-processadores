\section{Aritmética Binária e Somadores}
\setcounter{exercicio}{0}

\subsection{Complemento de Dois e Representação Numérica}

\exercicio{Representação de números negativos em complemento de dois}

Assuma vetores de 8 bits e codificação em complemento de dois.

\begin{enumerate}
  \item Escreva o valor em binário que representa $-5$.
  \item Escreva o valor em binário que representa $-230$.
  \item Qual valor está representado em binário? É positivo ou negativo? [10000011]
\end{enumerate}
Se algum valor não puder ser convertido, justifique.

\espacoresposta{3cm}

% --------------------------------------------------

\exercicio{Codificação ASCII}

Assuma vetores de 8 bits.

Escreva o valor em binário da letra \texttt{`G`} codificada em ASCII.

\espacoresposta{3cm}

% --------------------------------------------------
\subsection{Operações Aritméticas Binárias}

\exercicio{Soma binária com complemento de dois}

Faça a operação de soma binária a seguir e indique qual valor resulta
em decimal.

\[
\begin{array}{c}
10010111 \\
+ \\
01100010
\end{array}
\]

\espacoresposta{3cm}

% --------------------------------------------------

\exercicio{Soma binária com overflow}

Faça a operação de soma binária a seguir e indique qual valor resulta
em decimal.

\[
\begin{array}{c}
01111111 \\
+ \\
01100010
\end{array}
\]

\espacoresposta{3cm}

% --------------------------------------------------
\subsection{Somadores}

\exercicio{Implementação de um Half-Adder}

Considere a tabela verdade a seguir:

\[
\begin{array}{c c|c c}
a & b & soma & carry \\
\hline
0 & 0 & 0 & 0 \\
0 & 1 & 1 & 0 \\
1 & 0 & 1 & 0 \\
1 & 1 & 0 & 1
\end{array}
\]

Implemente um circuito Half-Adder utilizando portas lógicas.

\espacoresposta{3cm}

% --------------------------------------------------

\newpage
\exercicio{Implementação de um Full-Adder}

Considere a tabela verdade a seguir:

\[
\begin{array}{c c c|c c}
a & b & c_{in} & soma & carry \\
\hline
0 & 0 & 0 & 0 & 0 \\
0 & 0 & 1 & 1 & 0 \\
0 & 1 & 0 & 1 & 0 \\
0 & 1 & 1 & 0 & 1 \\
1 & 0 & 0 & 1 & 0 \\
1 & 0 & 1 & 0 & 1 \\
1 & 1 & 0 & 0 & 1 \\
1 & 1 & 1 & 1 & 1
\end{array}
\]

Implemente um circuito Full-Adder utilizando portas lógicas.

\espacoresposta{5cm}

% --------------------------------------------------

\exercicio{Somador de 2 bits}

Utilizando dois Full-Adders, implemente um circuito capaz de somar dois
vetores de 2 bits:

\[
X[1:0] + Y[1:0] \rightarrow S[1:0]
\]

Indique também o sinal de \textit{carry} de saída.

\newpage