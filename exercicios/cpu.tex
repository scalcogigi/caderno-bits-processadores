\section{Arquitetura, Execução e Debug da CPU}
\setcounter{exercicio}{0}

\subsection{Arquitetura da CPU}

\exercicio{Extensão da arquitetura da CPU}

Proponha uma modificação na CPU do Z01.1 que atenda aos seguintes requisitos:

\begin{enumerate}
  \item Adicionar um novo registrador \texttt{\%S}. Indique onde ele deve ser
  inserido na arquitetura.
  \item Modificar a linguagem de máquina do hardware para suportar esse novo
  registrador. Descreva a solução proposta.
  \item Possibilitar que o registrador \texttt{\%D} enderece a memória, permitindo
  a instrução:
  \[
  \texttt{movw \%A, (\%D)}
  \]
  \item Possibilitar carregamento imediato em \texttt{\%D}, permitindo a instrução:
  \[
  \texttt{leaw \$5, \%D}
  \]
\end{enumerate}

Para cada modificação, desenhe o diagrama da nova CPU.

\espacoresposta{5cm}

% --------------------------------------------------

\exercicio{Instrução \texttt{nop}}

Explique como a unidade de controle da CPU deve atuar para realizar a
instrução \texttt{nop}, detalhando os sinais de controle envolvidos.

\espacoresposta{3cm}

% --------------------------------------------------

\exercicio{Execução simultânea de instruções}

Analise se a CPU é capaz de executar, em um mesmo ciclo de clock,
as instruções \texttt{movw \%D, \%A} e \texttt{jg \%D}. Justifique sua
resposta com base na arquitetura da CPU, no caminho de dados
(\textit{datapath}) e nos sinais de controle envolvidos.

\espacoresposta{3cm}

% --------------------------------------------------

\exercicio{Sinal \texttt{loadPC}}

Explique quais sinais influenciam o carregamento do contador de programa
(\texttt{PC}) e como esse sinal afeta o fluxo de execução do programa.

\espacoresposta{3cm}

% ==================================================
\subsection{Execução e Debug da CPU}

\exercicio{Análise de falha em execução de programa}

Durante o desenvolvimento do projeto, o teste de integração
\texttt{./testeAssemblyMyCpu.py} falhou. A simulação executa o seguinte
programa:

\begin{verbatim}
leaw $0, %A
movw (%A), %D
leaw $1, %A
movw (%A), %A
addw %A, %D, (%A)
\end{verbatim}

A memória RAM inicial contém:

\begin{center}
\begin{tabular}{c c}
\toprule
Endereço & Dado \\
\midrule
0 & 0000000000000010 \\
1 & 0000000001000010 \\
3 & 0000000000000000 \\
\bottomrule
\end{tabular}
\end{center}

Com base na forma de onda dos sinais da CPU, identifique qual falha está
ocorrendo durante a execução do programa e explique o motivo.

\espacoresposta{3cm}

% --------------------------------------------------

\exercicio{Engenharia reversa de execução}

Você foi chamado para realizar engenharia reversa em um programa executando
no hardware do Z01. Na captura de dados, o sinal da instrução não foi obtido.

A partir da análise da forma de onda dos sinais da CPU, determine quais são
os \textbf{três comandos em assembly} que estão sendo executados no hardware
nesse momento.

\newpage