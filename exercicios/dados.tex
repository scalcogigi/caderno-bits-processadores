\section{Dados Digitais}
\setcounter{exercicio}{0}
\subsection{Representação de Dados e Memória}

\exercicio{Memória ROM — interpretação de dados}

Dois alunos empreendendo uso de engenharia reversa (para obter uma senha que
está gravada na memória) obtiveram uma série de “fotografias” da memória ROM
em um processador de 8 bits. Aplicando o conhecimento que obtiveram em robótica,
analisaram as imagens e extraíram a tabela a seguir, onde:
\[
\blacksquare = 1 \qquad \square = 0
\]

\begin{figure}[H]
    \centering
    \includegraphics[width=0.5\linewidth]{fotos/quadrados.png}
    \label{fig:placeholder}
\end{figure}

Ambos desejavam descobrir onde o programa armazena a senha de acesso (um número
que o usuário digita). Analisando o código-fonte do programa, descobriram que a
senha é armazenada no endereço \texttt{0x0C} da memória.

\textbf{Pergunta:} qual é o valor armazenado nesse endereço?

\espacoresposta{3cm}

% --------------------------------------------------

\exercicio{Comunicação serial — decodificação ASCII}

Você está debugando uma comunicação serial entre dois equipamentos e foi capaz
de capturar a seguinte sequência de dados, representada na forma de onda abaixo.

\begin{figure}[H]
    \centering
    \includegraphics[width=1\linewidth]{fotos/mensagem.png}
    \label{fig:placeholder}
\end{figure}

Sabendo que a mensagem transmitida está codificada em ASCII:

\textbf{Pergunta:} qual foi a mensagem enviada?

\espacoresposta{3cm}


% ==================================================
\subsection{Displays de Sete Segmentos}

\exercicio{Interpretação de sinais em display de 7 segmentos}

O display de sete segmentos é um componente muito utilizado em sistemas
eletrônicos para exibir dígitos numéricos:

\begin{figure}[H]
    \centering
    \includegraphics[width=0.5\linewidth]{fotos/7seg.png}
    \label{fig:placeholder}
\end{figure}

Por exemplo, para exibir o valor 7 é necessário acionar os segmentos
\texttt{a}, \texttt{b} e \texttt{c}, resultando na codificação:

\begin{figure}[H]
    \centering
    \includegraphics[width=0.25\linewidth]{fotos/sete.png}
    \caption{http://www.uize.com/examples/seven-segment-display.html}
    \label{fig:placeholder}
\end{figure}

Os segmentos normalmente formam um vetor do tipo g, f, e, d, c, b, a. O valor 7 seria codificado em: 0b0000111 -> 0x05

Você está fazendo um projeto que possui um display de 7 segmentos, mas ele está
queimado. Para descobrir quais valores estavam sendo enviados ao display, você
conectou um analisador de sinais nas trilhas correspondentes e obteve a forma de
onda apresentada.

\begin{figure}[H]
    \centering
    \includegraphics[width=1\linewidth]{fotos/ondas.png}
    \label{fig:placeholder}
\end{figure}

\textbf{Pergunta:} quais são os três valores (fachas) que estariam
sendo exibidos no display?

\espacoresposta{3cm}

% ==================================================
\subsection{Bases Numéricas e Codificações}

\exercicio{Tamanho de palavras em sistemas digitais}

Quantos bits possuem os seguintes tipos de dados?

\begin{itemize}
  \item nibble
  \item byte
  \item halfword
  \item word
\end{itemize}

\espacoresposta{0,5cm}

% --------------------------------------------------

\exercicio{Conversão de decimal para binário}

Converta os números decimais a seguir para binário, indicando a quantidade
mínima de bits necessária para cada representação.

\begin{center}
\begin{tabular}{|c|c|c|}
\hline
\textbf{Decimal} & \textbf{nº de bits} & \textbf{Binário} \\ \hline
0   &  &  \\
1   &  &  \\
2   &  &  \\
3   &  &  \\
4   &  &  \\
5   &  &  \\
6   &  &  \\
7   &  &  \\
8   &  &  \\
9   &  &  \\
10  &  &  \\
115 &  &  \\
256 &  &  \\
1027 & &  \\ \hline
\end{tabular}
\end{center}


% --------------------------------------------------

\exercicio{Conversão de binário para decimal}

Converta os seguintes números binários para decimal.

\begin{center}
\begin{tabular}{|c|c|}
\hline
\textbf{Binário} & \textbf{Decimal} \\ \hline
$0b0$      &  \\
$0b100$    &  \\
$0b10011$  &  \\
$0b11111$  &  \\
$0b01010$  &  \\ \hline
\end{tabular}
\end{center}


% --------------------------------------------------

\exercicio{Conversão de decimal para hexadecimal}

Converta os seguintes números decimais para hexadecimal.

\begin{center}
\begin{tabular}{|c|c|}
\hline
\textbf{Decimal} & \textbf{Hexadecimal} \\ \hline
0  &  \\
1  &  \\
2  &  \\
3  &  \\
4  &  \\
5  &  \\
6  &  \\
7  &  \\
8  &  \\
9  &  \\
10 &  \\
11 &  \\
12 &  \\
13 &  \\
14 &  \\
15 &  \\
16 &  \\
17 &  \\
18 &  \\ \hline
\end{tabular}
\end{center}


% --------------------------------------------------

\exercicio{Conversão de binário para hexadecimal}

Converta os seguintes números binários para hexadecimal.

\begin{center}
\begin{tabular}{|c|c|}
\hline
\textbf{Binário} & \textbf{Hexadecimal} \\ \hline
$0b11110$     &  \\
$0b101$       &  \\
$0b10100011$  &  \\
$0b11010$     &  \\
$0b00000$     &  \\ \hline
\end{tabular}
\end{center}


% --------------------------------------------------

\exercicio{Conversão de hexadecimal para decimal}

Converta os seguintes números hexadecimais para decimal.

\begin{center}
\begin{tabular}{|c|c|}
\hline
\textbf{Hexadecimal} & \textbf{Decimal} \\ \hline
$0x0003$ &  \\
$0xA$    &  \\
$0x55$   &  \\
$0x0101$ &  \\ \hline
\end{tabular}
\end{center}


% --------------------------------------------------

\exercicio{Conversão de decimal para octal}

Converta os seguintes números decimais para octal.

\begin{center}
\begin{tabular}{|c|c|}
\hline
\textbf{Decimal} & \textbf{Octal} \\ \hline
0  &  \\
1  &  \\
2  &  \\
3  &  \\
4  &  \\
5  &  \\
6  &  \\
7  &  \\
8  &  \\
9  &  \\
10 &  \\
11 &  \\
12 &  \\
13 &  \\
14 &  \\
15 &  \\
16 &  \\
17 &  \\ \hline
\end{tabular}
\end{center}


% --------------------------------------------------

\exercicio{Conversão de decimal para BCD}

Converta os seguintes números decimais para BCD.

\begin{center}
\begin{tabular}{|c|c|}
\hline
\textbf{Decimal} & \textbf{BCD} \\ \hline
50  &  \\
1   &  \\
103 &  \\
904 &  \\
4   &  \\
5   &  \\ \hline
\end{tabular}
\end{center}


% --------------------------------------------------

\exercicio{Conversão de caracteres para ASCII}

Faça a conversão dos seguintes caracteres para o código ASCII binário
correspondente.

\begin{center}
\begin{tabular}{|c|c|}
\hline
\textbf{Caractere} & \textbf{ASCII (binário)} \\ \hline
`1' &  \\
`a' &  \\
`A' &  \\
`Z' &  \\
`0' &  \\ \hline
\end{tabular}
\end{center}

\newpage