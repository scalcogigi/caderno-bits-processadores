\section{Álgebra Booleana}
\setcounter{exercicio}{0}

\subsection{Portas Lógicas e Tabelas Verdade}

\exercicio{Indique o nome e a operação booleana associada a cada elemento}

\begin{figure}[H]
\centering
\includegraphics[width=0.6\linewidth]{fotos/portas_logicas.png}
\end{figure}

\espacoresposta{2cm}

\exercicio{Identificação de portas lógicas a partir de tabelas verdade}

\textbf{(a)} A tabela representa qual porta lógica?

\[
\begin{array}{|c|c|c|}
\hline
A & B & OUT\\ \hline
0 & 0 & 0\\
0 & 1 & 1\\
1 & 0 & 1\\
1 & 1 & 0\\ \hline
\end{array}
\]
\textbf{(b)} A tabela representa qual porta lógica?

\[
\begin{array}{|c|c|c|}
\hline
A & B & OUT\\ \hline
0 & 0 & 1\\
0 & 1 & 1\\
1 & 0 & 1\\
1 & 1 & 0\\ \hline
\end{array}
\]

\espacoresposta{3cm}

\subsection{Propriedades da Álgebra Booleana}

\exercicio{Aplicando a lei da distributividade}

\[
A(B+\overline{C}+D)
\]

\begin{enumerate}
\item $AB + AC + AD$
\item $ABCD$
\item $A + B + C + D$
\item $AB + A\overline{C} + AD$
\end{enumerate}

\espacoresposta{0.5cm}

\exercicio{Aplicando o teorema de De Morgan}

\[
\overline{A + B + C}
\]

\begin{enumerate}
\item $\overline{A} + \overline{B} + \overline{C}$
\item $\overline{A+B+C}$
\item $A + \overline{B} + C$
\item $A(B+C)$
\end{enumerate}

\espacoresposta{0.5cm}

\exercicio{Indique quais simplificações estão incorretas.}

\begin{enumerate}
  \item $\overline{(x + y)} = \overline{x} \cdot \overline{y} = x \cdot y$
  \item $x(\overline{x} + y) = x \cdot \overline{x} + x \cdot y = 0 + x \cdot y = x \cdot y$
  \item $x \cdot y + x(y + z) = x \cdot y + x \cdot y + z = x \cdot y + z$
  \item $\overline{x} \cdot \overline{y} \cdot z + \overline{x} \cdot y \cdot z + x \cdot \overline{y}
        = \overline{x} \cdot z(\overline{y} + y) + x \cdot \overline{y}
        = \overline{x} \cdot z + x \cdot \overline{y}$
\end{enumerate}


\espacoresposta{0.5cm}

\exercicio{Indique qual propriedade da álgebra booleana é falsa}

\begin{enumerate}
\item $A(\overline{A}+B)=AB$
\item $A+(AB)=A$
\item $A+A=A$
\item $A\cdot A=A$
\end{enumerate}

\espacoresposta{0.5cm}

\subsection{Formas Canônicas}

\exercicio{Forma canônica Soma de Produtos}

\[
\begin{array}{|c|c|c|}
\hline
A & B & Q\\ \hline
0 & 0 & 1\\
0 & 1 & 0\\
1 & 0 & 0\\
1 & 1 & 1\\ \hline
\end{array}
\]

\begin{enumerate}
\item $\overline{A}\overline{B}+AB$
\item $\overline{A}B+A\overline{B}$
\item $\overline{A}\overline{B}+A\overline{B}$
\item $\overline{A}B+AB$
\end{enumerate}

\espacoresposta{0.5cm}

\exercicio{Dado a seguinte tabela verdade:}

\[
\begin{array}{cccc}
A & B & C & Q\\ \hline
0&0&0&1\\
0&0&1&0\\
0&1&0&0\\
0&1&1&1\\
1&0&0&1\\
1&0&1&0\\
1&1&0&0\\
1&1&1&1
\end{array}
\]

\begin{enumerate}
\item Crie uma fórmula em álgebra booleana que represente a tabela via SoP e PoS.
\item Simplifique a SoP.
\item Desenhe o circuito correspondente.
\end{enumerate}

\espacoresposta{4cm}

\exercicio{Contagem de saídas em uma expressão SoP}

\[
A\overline{BC} + \overline{A}BC + \overline{A}B\overline{C} +
A\overline{B}\overline{C} + ABC
\]

\espacoresposta{0.5cm}

\exercicio{Característica da forma canônica SoP}

\begin{enumerate}
\item Os circuitos lógicos são reduzidos a nada mais do que simples portas AND e OR
\item Os tempos de atraso são muito reduzidas em relação a outras formas
\item Nenhum sinal deve passar por mais de duas portas lógicas, não incluindo inversores
\item O número máximo de portas que qualquer sinal deve passar é reduzido por um factor de dois
\end{enumerate}

\espacoresposta{0.5cm}

\subsection{Mapas de Karnaugh}

\exercicio{Simplifique a expressão lógica utilizando álgebra booleana}

\[
\overline{A}BC + \overline{A}BC + \overline{A}BC +
ABC + ABC
\]

\espacoresposta{4cm}

\exercicio{Determine a expressão lógica simplificada para os mapas de Karnaugh}

\textbf{Mapa 1 (AB / C):}
\[
\begin{array}{c|cccc}
AB\backslash C & 00 & 01 & 11 & 10\\ \midrule
0 & 0 & 1 & 1 & 1 \\
1 & 0 & 1 & 1 & 1
\end{array}
\]

\vspace{0.5em}

\textbf{Mapa 2 (AB / C):}
\[
\begin{array}{c|cccc}
AB\backslash C & 00 & 01 & 11 & 10\\ \midrule
0 & 0 & 1 & 0 & 1 \\
1 & 1 & 0 & 0 & 1
\end{array}
\]

\vspace{0.5em}

\textbf{Mapa 3 (AB / CD):}
\[
\begin{array}{c|cccc}
AB\backslash CD & 00 & 01 & 11 & 10\\ \midrule
00 & 1 & 0 & 0 & 1\\
01 & 0 & 0 & 0 & 1\\
11 & 0 & 0 & 0 & 0\\
10 & 0 & 1 & 1 & 0
\end{array}
\]

\vspace{0.5em}

\textbf{Mapa 4 (AB / CD):}
\[
\begin{array}{c|cccc}
AB\backslash CD & 00 & 01 & 11 & 10\\ \midrule
00 & 1 & 0 & 1 & 1\\
01 & 0 & 0 & 1 & 1\\
11 & 0 & 0 & 1 & 1\\
10 & 1 & 1 & 1 & 1
\end{array}
\]

\espacoresposta{2cm}

\exercicio{Utilize um mapa de Karnaugh para simplificar a expressão}

\[
ABC\overline{D} + \overline{A}\overline{B}CD +
A\overline{B}\overline{C}D + \overline{A} + \overline{B} +
\overline{C} + \overline{D}
\]

\espacoresposta{3cm}

\exercicio{Conversão de Produto de Somas para Soma de Produtos}

A seguinte expressão foi resultado da forma canônica do Produto de Somas
de uma tabela verdade para a produção de um circuito lógico. Converta
essa expressão para Soma de Produtos e minimize.

\[
(A+B+C)(A+B+\overline{C})(A+\overline{B}+C)
(\overline{A}+B+C)(\overline{A}+\overline{B}+C)
\]

\newpage