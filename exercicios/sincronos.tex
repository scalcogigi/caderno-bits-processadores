\section{Sistemas Síncronos e Elementos de Memória}
\setcounter{exercicio}{0}

\subsection{Conceitos Fundamentais}

\exercicio{Diferença entre circuitos síncronos e assíncronos}

Explique qual é a diferença entre um circuito síncrono e um circuito
assíncrono, destacando o papel do tempo, do sinal de clock e da
sincronização dos sinais.

\espacoresposta{3cm}

% --------------------------------------------------

\exercicio{O papel do clock em um computador}

Explique qual é o papel do sinal de clock em um computador e por que
ele é fundamental para o funcionamento correto de sistemas digitais.

\espacoresposta{3cm}

% --------------------------------------------------

\exercicio{Latch versus Flip-Flop}

Explique qual é a diferença entre um latch e um flip-flop (FF),
considerando o comportamento em relação ao sinal de clock e o momento
em que a saída é atualizada.

\espacoresposta{3cm}

% --------------------------------------------------
\subsection{Flip-Flop Tipo D}

\exercicio{Funcionamento do Flip-Flop tipo D}

Explique o que é um Flip-Flop do tipo D e descreva como ele funciona
em relação às entradas de dados e ao sinal de clock.

\espacoresposta{3cm}

% --------------------------------------------------

\exercicio{Tabela verdade do Flip-Flop tipo D}

Preencha a tabela verdade do Flip-Flop tipo D a seguir, considerando
as entradas indicadas e o comportamento do sinal de clock.

\[
\begin{array}{c|c|c}
\text{Clock} & D & Q(t+1) \\ \hline
\uparrow & 0 & \\
\uparrow & 1 & \\
0/1 & X & Q(t)
\end{array}
\]

\espacoresposta{3cm}


% --------------------------------------------------
\exercicio{Análise temporal do sinal de saída}

Observe o diagrama de sinais apresentado a seguir (clock, D, preset e clear)
e determine qual será o valor da saída $Q$ nos instantes A, B, C, D e E (cada linha pontilhada)

\begin{center}
\includegraphics[width=0.7\linewidth]{fotos/diagramaD.png}
\end{center}

\espacoresposta{3cm}

% --------------------------------------------------
\subsection{Memória e Registradores}


\exercicio{O que é um registrador}

Explique o que é um registrador em um sistema digital e qual é a sua
função dentro da arquitetura de um computador.

\espacoresposta{3cm}

% --------------------------------------------------

\exercicio{Sinal de endereço em uma memória}

Explique para que serve o sinal de endereço em uma memória digital e
como ele está relacionado à organização e ao acesso aos dados armazenados.

\espacoresposta{3cm}

\exercicio{Binary Digit (Bit) ao longo do tempo}

Considere o Binary Digit apresentado na tabela a seguir e complete os campos
em branco, indicando a evolução do valor armazenado ao longo do tempo,
de acordo com os sinais de controle.

\[
\begin{array}{c|c|c}
\text{Clock} & \text{Entrada} & \text{Bit armazenado} \\ \hline
\uparrow & 0 & \\
\uparrow & 1 & \\
0/1 & X & \text{mantém valor}
\end{array}
\]

\espacoresposta{3cm}

\newpage