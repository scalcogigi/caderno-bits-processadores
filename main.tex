\documentclass[12pt,a4paper]{article}


\usepackage[utf8]{inputenc}
\usepackage[T1]{fontenc}
\usepackage[brazil]{babel}
\usepackage{amsmath,amssymb}
\usepackage{geometry}
\usepackage{graphicx}
\usepackage{float}
\usepackage{array}
\usepackage{booktabs}
\usepackage{xcolor}
\usepackage[hidelinks]{hyperref}
\usepackage{tikz}
\usetikzlibrary{circuits.logic.US, positioning, arrows.meta}
\usepackage{circuitikz}



\geometry{margin=2cm}

\definecolor{roxoclaro}{RGB}{150,90,180}


\newcommand{\espacoresposta}[1]{%
\vspace{#1}
\noindent\textcolor{roxoclaro}{\rule{\linewidth}{0.6pt}}
}

\newcounter{exercicio}

\newcommand{\exercicio}[1]{%
  \refstepcounter{exercicio}%
  \subsubsection*{Exercício \theexercicio\ --- #1}%
}


\newcommand{\gabaritoex}[1]{%
\subsection*{Exercício #1}
}


% ====================
% Pacotes
% ====================
\usepackage{setspace}
\usepackage{titlesec}
\usepackage[hidelinks]{hyperref}
\usepackage{tocloft}

% ====================
% Cores
% ====================
\definecolor{roxoclaro}{RGB}{150,90,180}
\definecolor{cinza}{RGB}{90,90,90}
\definecolor{amareloclaro}{RGB}{240,200,90}

% ====================
% Profundidade do sumário
% ====================
\setcounter{tocdepth}{2}

% ====================
% Estilo dos títulos
% ====================
\titleformat{\section}
  {\color{roxoclaro}\Large\bfseries}
  {}
  {0pt}
  {}

\titleformat{\subsection}
  {\color{roxoclaro}\large\bfseries}
  {}
  {0pt}
  {}

% ====================
% Estilo do sumário
% ====================
\renewcommand{\cftsecfont}{\color{roxoclaro}\bfseries}
\renewcommand{\cftsubsecfont}{\color{cinza}}
\renewcommand{\cftsecpagefont}{\color{cinza}}
\renewcommand{\cftsubsecpagefont}{\color{cinza}}

% ====================
% Capa
% ====================
\title{
\vspace{-1.5cm}
{\Huge\bfseries Caderno de Exercícios}\\[0.4cm]
{\Large\color{roxoclaro}Bits e Processadores}\\[0.6cm]
\color{roxoclaro}\rule{0.6\linewidth}{0.8pt}
}

\author{
{\large\color{cinza}Professor: Rafael Corsi Ferrão}\\[0.2cm]
{\large\color{cinza}Ninja: Giovanna Barros Scalco}
}

\date{{\color{cinza}2025}}

% ====================
\begin{document}
\maketitle
\thispagestyle{empty}

\vspace{1cm}

% ====================
% Apresentação
% ====================
\begin{center}
\begin{minipage}{0.85\linewidth}
\setstretch{1.2}

{\Large\color{roxoclaro}\bfseries Apresentação}

\vspace{0.6cm}

{\color{cinza}
Este caderno reúne exercícios referentes aos conteúdos da disciplina
\textbf{Bits e Processadores}. O material foi organizado de forma progressiva,
com foco na construção do raciocínio lógico, na compreensão conceitual e
na relação entre teoria e prática em sistemas digitais.

\vspace{0.4cm}

A proposta é que este caderno funcione como um espaço de estudo ativo:
anotações, tentativas, erros e correções fazem parte do processo.
Os exercícios foram pensados para incentivar a interpretação, a análise
e a autonomia na resolução de problemas.

\vspace{0.4cm}

\textbf{Sugestão de uso:} tente resolver cada exercício antes de consultar
o gabarito comentado ao final do material. O objetivo não é apenas chegar
à resposta correta, mas compreender o caminho até ela.
}

\end{minipage}
\end{center}

\newpage

% ====================
% Sumário
% ====================
\tableofcontents

\newpage

% ====================
% Exercícios
% ====================
\section*{Exercícios}
\addcontentsline{toc}{section}{Exercícios}

\section{Álgebra Booleana}
\setcounter{exercicio}{0}

\subsection{Portas Lógicas e Tabelas Verdade}

\exercicio{Indique o nome e a operação booleana associada a cada elemento}

\begin{figure}[H]
\centering
\includegraphics[width=0.6\linewidth]{fotos/portas_logicas.png}
\end{figure}

\espacoresposta{2cm}

\exercicio{Identificação de portas lógicas a partir de tabelas verdade}

\textbf{(a)} A tabela representa qual porta lógica?

\[
\begin{array}{|c|c|c|}
\hline
A & B & OUT\\ \hline
0 & 0 & 0\\
0 & 1 & 1\\
1 & 0 & 1\\
1 & 1 & 0\\ \hline
\end{array}
\]
\textbf{(b)} A tabela representa qual porta lógica?

\[
\begin{array}{|c|c|c|}
\hline
A & B & OUT\\ \hline
0 & 0 & 1\\
0 & 1 & 1\\
1 & 0 & 1\\
1 & 1 & 0\\ \hline
\end{array}
\]

\espacoresposta{3cm}

\subsection{Propriedades da Álgebra Booleana}

\exercicio{Aplicando a lei da distributividade}

\[
A(B+\overline{C}+D)
\]

\begin{enumerate}
\item $AB + AC + AD$
\item $ABCD$
\item $A + B + C + D$
\item $AB + A\overline{C} + AD$
\end{enumerate}

\espacoresposta{0.5cm}

\exercicio{Aplicando o teorema de De Morgan}

\[
\overline{A + B + C}
\]

\begin{enumerate}
\item $\overline{A} + \overline{B} + \overline{C}$
\item $\overline{A+B+C}$
\item $A + \overline{B} + C$
\item $A(B+C)$
\end{enumerate}

\espacoresposta{0.5cm}

\exercicio{Indique quais simplificações estão incorretas.}

\begin{enumerate}
  \item $\overline{(x + y)} = \overline{x} \cdot \overline{y} = x \cdot y$
  \item $x(\overline{x} + y) = x \cdot \overline{x} + x \cdot y = 0 + x \cdot y = x \cdot y$
  \item $x \cdot y + x(y + z) = x \cdot y + x \cdot y + z = x \cdot y + z$
  \item $\overline{x} \cdot \overline{y} \cdot z + \overline{x} \cdot y \cdot z + x \cdot \overline{y}
        = \overline{x} \cdot z(\overline{y} + y) + x \cdot \overline{y}
        = \overline{x} \cdot z + x \cdot \overline{y}$
\end{enumerate}


\espacoresposta{0.5cm}

\exercicio{Indique qual propriedade da álgebra booleana é falsa}

\begin{enumerate}
\item $A(\overline{A}+B)=AB$
\item $A+(AB)=A$
\item $A+A=A$
\item $A\cdot A=A$
\end{enumerate}

\espacoresposta{0.5cm}

\subsection{Formas Canônicas}

\exercicio{Forma canônica Soma de Produtos}

\[
\begin{array}{|c|c|c|}
\hline
A & B & Q\\ \hline
0 & 0 & 1\\
0 & 1 & 0\\
1 & 0 & 0\\
1 & 1 & 1\\ \hline
\end{array}
\]

\begin{enumerate}
\item $\overline{A}\overline{B}+AB$
\item $\overline{A}B+A\overline{B}$
\item $\overline{A}\overline{B}+A\overline{B}$
\item $\overline{A}B+AB$
\end{enumerate}

\espacoresposta{0.5cm}

\exercicio{Dado a seguinte tabela verdade:}

\[
\begin{array}{cccc}
A & B & C & Q\\ \hline
0&0&0&1\\
0&0&1&0\\
0&1&0&0\\
0&1&1&1\\
1&0&0&1\\
1&0&1&0\\
1&1&0&0\\
1&1&1&1
\end{array}
\]

\begin{enumerate}
\item Crie uma fórmula em álgebra booleana que represente a tabela via SoP e PoS.
\item Simplifique a SoP.
\item Desenhe o circuito correspondente.
\end{enumerate}

\espacoresposta{4cm}

\exercicio{Contagem de saídas em uma expressão SoP}

\[
A\overline{BC} + \overline{A}BC + \overline{A}B\overline{C} +
A\overline{B}\overline{C} + ABC
\]

\espacoresposta{0.5cm}

\exercicio{Característica da forma canônica SoP}

\begin{enumerate}
\item Os circuitos lógicos são reduzidos a nada mais do que simples portas AND e OR
\item Os tempos de atraso são muito reduzidas em relação a outras formas
\item Nenhum sinal deve passar por mais de duas portas lógicas, não incluindo inversores
\item O número máximo de portas que qualquer sinal deve passar é reduzido por um factor de dois
\end{enumerate}

\espacoresposta{0.5cm}

\subsection{Mapas de Karnaugh}

\exercicio{Simplifique a expressão lógica utilizando álgebra booleana}

\[
\overline{A}BC + \overline{A}BC + \overline{A}BC +
ABC + ABC
\]

\espacoresposta{4cm}

\exercicio{Determine a expressão lógica simplificada para os mapas de Karnaugh}

\textbf{Mapa 1 (AB / C):}
\[
\begin{array}{c|cccc}
AB\backslash C & 00 & 01 & 11 & 10\\ \midrule
0 & 0 & 1 & 1 & 1 \\
1 & 0 & 1 & 1 & 1
\end{array}
\]

\vspace{0.5em}

\textbf{Mapa 2 (AB / C):}
\[
\begin{array}{c|cccc}
AB\backslash C & 00 & 01 & 11 & 10\\ \midrule
0 & 0 & 1 & 0 & 1 \\
1 & 1 & 0 & 0 & 1
\end{array}
\]

\vspace{0.5em}

\textbf{Mapa 3 (AB / CD):}
\[
\begin{array}{c|cccc}
AB\backslash CD & 00 & 01 & 11 & 10\\ \midrule
00 & 1 & 0 & 0 & 1\\
01 & 0 & 0 & 0 & 1\\
11 & 0 & 0 & 0 & 0\\
10 & 0 & 1 & 1 & 0
\end{array}
\]

\vspace{0.5em}

\textbf{Mapa 4 (AB / CD):}
\[
\begin{array}{c|cccc}
AB\backslash CD & 00 & 01 & 11 & 10\\ \midrule
00 & 1 & 0 & 1 & 1\\
01 & 0 & 0 & 1 & 1\\
11 & 0 & 0 & 1 & 1\\
10 & 1 & 1 & 1 & 1
\end{array}
\]

\espacoresposta{2cm}

\exercicio{Utilize um mapa de Karnaugh para simplificar a expressão}

\[
ABC\overline{D} + \overline{A}\overline{B}CD +
A\overline{B}\overline{C}D + \overline{A} + \overline{B} +
\overline{C} + \overline{D}
\]

\espacoresposta{3cm}

\exercicio{Conversão de Produto de Somas para Soma de Produtos}

A seguinte expressão foi resultado da forma canônica do Produto de Somas
de uma tabela verdade para a produção de um circuito lógico. Converta
essa expressão para Soma de Produtos e minimize.

\[
(A+B+C)(A+B+\overline{C})(A+\overline{B}+C)
(\overline{A}+B+C)(\overline{A}+\overline{B}+C)
\]

\newpage
\newpage
\section{Dados Digitais}
\setcounter{exercicio}{0}
\subsection{Representação de Dados e Memória}

\exercicio{Memória ROM — interpretação de dados}

Dois alunos empreendendo uso de engenharia reversa (para obter uma senha que
está gravada na memória) obtiveram uma série de “fotografias” da memória ROM
em um processador de 8 bits. Aplicando o conhecimento que obtiveram em robótica,
analisaram as imagens e extraíram a tabela a seguir, onde:
\[
\blacksquare = 1 \qquad \square = 0
\]

\begin{figure}[H]
    \centering
    \includegraphics[width=0.5\linewidth]{fotos/quadrados.png}
    \label{fig:placeholder}
\end{figure}

Ambos desejavam descobrir onde o programa armazena a senha de acesso (um número
que o usuário digita). Analisando o código-fonte do programa, descobriram que a
senha é armazenada no endereço \texttt{0x0C} da memória.

\textbf{Pergunta:} qual é o valor armazenado nesse endereço?

\espacoresposta{3cm}

% --------------------------------------------------

\exercicio{Comunicação serial — decodificação ASCII}

Você está debugando uma comunicação serial entre dois equipamentos e foi capaz
de capturar a seguinte sequência de dados, representada na forma de onda abaixo.

\begin{figure}[H]
    \centering
    \includegraphics[width=1\linewidth]{fotos/mensagem.png}
    \label{fig:placeholder}
\end{figure}

Sabendo que a mensagem transmitida está codificada em ASCII:

\textbf{Pergunta:} qual foi a mensagem enviada?

\espacoresposta{3cm}


% ==================================================
\subsection{Displays de Sete Segmentos}

\exercicio{Interpretação de sinais em display de 7 segmentos}

O display de sete segmentos é um componente muito utilizado em sistemas
eletrônicos para exibir dígitos numéricos:

\begin{figure}[H]
    \centering
    \includegraphics[width=0.5\linewidth]{fotos/7seg.png}
    \label{fig:placeholder}
\end{figure}

Por exemplo, para exibir o valor 7 é necessário acionar os segmentos
\texttt{a}, \texttt{b} e \texttt{c}, resultando na codificação:

\begin{figure}[H]
    \centering
    \includegraphics[width=0.25\linewidth]{fotos/sete.png}
    \caption{http://www.uize.com/examples/seven-segment-display.html}
    \label{fig:placeholder}
\end{figure}

Os segmentos normalmente formam um vetor do tipo g, f, e, d, c, b, a. O valor 7 seria codificado em: 0b0000111 -> 0x05

Você está fazendo um projeto que possui um display de 7 segmentos, mas ele está
queimado. Para descobrir quais valores estavam sendo enviados ao display, você
conectou um analisador de sinais nas trilhas correspondentes e obteve a forma de
onda apresentada.

\begin{figure}[H]
    \centering
    \includegraphics[width=1\linewidth]{fotos/ondas.png}
    \label{fig:placeholder}
\end{figure}

\textbf{Pergunta:} quais são os três valores (fachas) que estariam
sendo exibidos no display?

\espacoresposta{3cm}

% ==================================================
\subsection{Bases Numéricas e Codificações}

\exercicio{Tamanho de palavras em sistemas digitais}

Quantos bits possuem os seguintes tipos de dados?

\begin{itemize}
  \item nibble
  \item byte
  \item halfword
  \item word
\end{itemize}

\espacoresposta{0,5cm}

% --------------------------------------------------

\exercicio{Conversão de decimal para binário}

Converta os números decimais a seguir para binário, indicando a quantidade
mínima de bits necessária para cada representação.

\begin{center}
\begin{tabular}{|c|c|c|}
\hline
\textbf{Decimal} & \textbf{nº de bits} & \textbf{Binário} \\ \hline
0   &  &  \\
1   &  &  \\
2   &  &  \\
3   &  &  \\
4   &  &  \\
5   &  &  \\
6   &  &  \\
7   &  &  \\
8   &  &  \\
9   &  &  \\
10  &  &  \\
115 &  &  \\
256 &  &  \\
1027 & &  \\ \hline
\end{tabular}
\end{center}


% --------------------------------------------------

\exercicio{Conversão de binário para decimal}

Converta os seguintes números binários para decimal.

\begin{center}
\begin{tabular}{|c|c|}
\hline
\textbf{Binário} & \textbf{Decimal} \\ \hline
$0b0$      &  \\
$0b100$    &  \\
$0b10011$  &  \\
$0b11111$  &  \\
$0b01010$  &  \\ \hline
\end{tabular}
\end{center}


% --------------------------------------------------

\exercicio{Conversão de decimal para hexadecimal}

Converta os seguintes números decimais para hexadecimal.

\begin{center}
\begin{tabular}{|c|c|}
\hline
\textbf{Decimal} & \textbf{Hexadecimal} \\ \hline
0  &  \\
1  &  \\
2  &  \\
3  &  \\
4  &  \\
5  &  \\
6  &  \\
7  &  \\
8  &  \\
9  &  \\
10 &  \\
11 &  \\
12 &  \\
13 &  \\
14 &  \\
15 &  \\
16 &  \\
17 &  \\
18 &  \\ \hline
\end{tabular}
\end{center}


% --------------------------------------------------

\exercicio{Conversão de binário para hexadecimal}

Converta os seguintes números binários para hexadecimal.

\begin{center}
\begin{tabular}{|c|c|}
\hline
\textbf{Binário} & \textbf{Hexadecimal} \\ \hline
$0b11110$     &  \\
$0b101$       &  \\
$0b10100011$  &  \\
$0b11010$     &  \\
$0b00000$     &  \\ \hline
\end{tabular}
\end{center}


% --------------------------------------------------

\exercicio{Conversão de hexadecimal para decimal}

Converta os seguintes números hexadecimais para decimal.

\begin{center}
\begin{tabular}{|c|c|}
\hline
\textbf{Hexadecimal} & \textbf{Decimal} \\ \hline
$0x0003$ &  \\
$0xA$    &  \\
$0x55$   &  \\
$0x0101$ &  \\ \hline
\end{tabular}
\end{center}


% --------------------------------------------------

\exercicio{Conversão de decimal para octal}

Converta os seguintes números decimais para octal.

\begin{center}
\begin{tabular}{|c|c|}
\hline
\textbf{Decimal} & \textbf{Octal} \\ \hline
0  &  \\
1  &  \\
2  &  \\
3  &  \\
4  &  \\
5  &  \\
6  &  \\
7  &  \\
8  &  \\
9  &  \\
10 &  \\
11 &  \\
12 &  \\
13 &  \\
14 &  \\
15 &  \\
16 &  \\
17 &  \\ \hline
\end{tabular}
\end{center}


% --------------------------------------------------

\exercicio{Conversão de decimal para BCD}

Converta os seguintes números decimais para BCD.

\begin{center}
\begin{tabular}{|c|c|}
\hline
\textbf{Decimal} & \textbf{BCD} \\ \hline
50  &  \\
1   &  \\
103 &  \\
904 &  \\
4   &  \\
5   &  \\ \hline
\end{tabular}
\end{center}


% --------------------------------------------------

\exercicio{Conversão de caracteres para ASCII}

Faça a conversão dos seguintes caracteres para o código ASCII binário
correspondente.

\begin{center}
\begin{tabular}{|c|c|}
\hline
\textbf{Caractere} & \textbf{ASCII (binário)} \\ \hline
`1' &  \\
`a' &  \\
`A' &  \\
`Z' &  \\
`0' &  \\ \hline
\end{tabular}
\end{center}

\newpage
\newpage
\section{Sistemas Síncronos e Elementos de Memória}
\setcounter{exercicio}{0}

\subsection{Conceitos Fundamentais}

\exercicio{Diferença entre circuitos síncronos e assíncronos}

Explique qual é a diferença entre um circuito síncrono e um circuito
assíncrono, destacando o papel do tempo, do sinal de clock e da
sincronização dos sinais.

\espacoresposta{3cm}

% --------------------------------------------------

\exercicio{O papel do clock em um computador}

Explique qual é o papel do sinal de clock em um computador e por que
ele é fundamental para o funcionamento correto de sistemas digitais.

\espacoresposta{3cm}

% --------------------------------------------------

\exercicio{Latch versus Flip-Flop}

Explique qual é a diferença entre um latch e um flip-flop (FF),
considerando o comportamento em relação ao sinal de clock e o momento
em que a saída é atualizada.

\espacoresposta{3cm}

% --------------------------------------------------
\subsection{Flip-Flop Tipo D}

\exercicio{Funcionamento do Flip-Flop tipo D}

Explique o que é um Flip-Flop do tipo D e descreva como ele funciona
em relação às entradas de dados e ao sinal de clock.

\espacoresposta{3cm}

% --------------------------------------------------

\exercicio{Tabela verdade do Flip-Flop tipo D}

Preencha a tabela verdade do Flip-Flop tipo D a seguir, considerando
as entradas indicadas e o comportamento do sinal de clock.

\[
\begin{array}{c|c|c}
\text{Clock} & D & Q(t+1) \\ \hline
\uparrow & 0 & \\
\uparrow & 1 & \\
0/1 & X & Q(t)
\end{array}
\]

\espacoresposta{3cm}


% --------------------------------------------------
\exercicio{Análise temporal do sinal de saída}

Observe o diagrama de sinais apresentado a seguir (clock, D, preset e clear)
e determine qual será o valor da saída $Q$ nos instantes A, B, C, D e E (cada linha pontilhada)

\begin{center}
\includegraphics[width=0.7\linewidth]{fotos/diagramaD.png}
\end{center}

\espacoresposta{3cm}

% --------------------------------------------------
\subsection{Memória e Registradores}


\exercicio{O que é um registrador}

Explique o que é um registrador em um sistema digital e qual é a sua
função dentro da arquitetura de um computador.

\espacoresposta{3cm}

% --------------------------------------------------

\exercicio{Sinal de endereço em uma memória}

Explique para que serve o sinal de endereço em uma memória digital e
como ele está relacionado à organização e ao acesso aos dados armazenados.

\espacoresposta{3cm}

\exercicio{Binary Digit (Bit) ao longo do tempo}

Considere o Binary Digit apresentado na tabela a seguir e complete os campos
em branco, indicando a evolução do valor armazenado ao longo do tempo,
de acordo com os sinais de controle.

\[
\begin{array}{c|c|c}
\text{Clock} & \text{Entrada} & \text{Bit armazenado} \\ \hline
\uparrow & 0 & \\
\uparrow & 1 & \\
0/1 & X & \text{mantém valor}
\end{array}
\]

\espacoresposta{3cm}

\newpage
\newpage
\section{Aritmética Binária e Somadores}
\setcounter{exercicio}{0}

\subsection{Complemento de Dois e Representação Numérica}

\exercicio{Representação de números negativos em complemento de dois}

Assuma vetores de 8 bits e codificação em complemento de dois.

\begin{enumerate}
  \item Escreva o valor em binário que representa $-5$.
  \item Escreva o valor em binário que representa $-230$.
  \item Qual valor está representado em binário? É positivo ou negativo? [10000011]
\end{enumerate}
Se algum valor não puder ser convertido, justifique.

\espacoresposta{3cm}

% --------------------------------------------------

\exercicio{Codificação ASCII}

Assuma vetores de 8 bits.

Escreva o valor em binário da letra \texttt{`G`} codificada em ASCII.

\espacoresposta{3cm}

% --------------------------------------------------
\subsection{Operações Aritméticas Binárias}

\exercicio{Soma binária com complemento de dois}

Faça a operação de soma binária a seguir e indique qual valor resulta
em decimal.

\[
\begin{array}{c}
10010111 \\
+ \\
01100010
\end{array}
\]

\espacoresposta{3cm}

% --------------------------------------------------

\exercicio{Soma binária com overflow}

Faça a operação de soma binária a seguir e indique qual valor resulta
em decimal.

\[
\begin{array}{c}
01111111 \\
+ \\
01100010
\end{array}
\]

\espacoresposta{3cm}

% --------------------------------------------------
\subsection{Somadores}

\exercicio{Implementação de um Half-Adder}

Considere a tabela verdade a seguir:

\[
\begin{array}{c c|c c}
a & b & soma & carry \\
\hline
0 & 0 & 0 & 0 \\
0 & 1 & 1 & 0 \\
1 & 0 & 1 & 0 \\
1 & 1 & 0 & 1
\end{array}
\]

Implemente um circuito Half-Adder utilizando portas lógicas.

\espacoresposta{3cm}

% --------------------------------------------------

\newpage
\exercicio{Implementação de um Full-Adder}

Considere a tabela verdade a seguir:

\[
\begin{array}{c c c|c c}
a & b & c_{in} & soma & carry \\
\hline
0 & 0 & 0 & 0 & 0 \\
0 & 0 & 1 & 1 & 0 \\
0 & 1 & 0 & 1 & 0 \\
0 & 1 & 1 & 0 & 1 \\
1 & 0 & 0 & 1 & 0 \\
1 & 0 & 1 & 0 & 1 \\
1 & 1 & 0 & 0 & 1 \\
1 & 1 & 1 & 1 & 1
\end{array}
\]

Implemente um circuito Full-Adder utilizando portas lógicas.

\espacoresposta{5cm}

% --------------------------------------------------

\exercicio{Somador de 2 bits}

Utilizando dois Full-Adders, implemente um circuito capaz de somar dois
vetores de 2 bits:

\[
X[1:0] + Y[1:0] \rightarrow S[1:0]
\]

Indique também o sinal de \textit{carry} de saída.

\newpage
\newpage
\section{Arquitetura, Execução e Debug da CPU}
\setcounter{exercicio}{0}

\subsection{Arquitetura da CPU}

\exercicio{Extensão da arquitetura da CPU}

Proponha uma modificação na CPU do Z01.1 que atenda aos seguintes requisitos:

\begin{enumerate}
  \item Adicionar um novo registrador \texttt{\%S}. Indique onde ele deve ser
  inserido na arquitetura.
  \item Modificar a linguagem de máquina do hardware para suportar esse novo
  registrador. Descreva a solução proposta.
  \item Possibilitar que o registrador \texttt{\%D} enderece a memória, permitindo
  a instrução:
  \[
  \texttt{movw \%A, (\%D)}
  \]
  \item Possibilitar carregamento imediato em \texttt{\%D}, permitindo a instrução:
  \[
  \texttt{leaw \$5, \%D}
  \]
\end{enumerate}

Para cada modificação, desenhe o diagrama da nova CPU.

\espacoresposta{5cm}

% --------------------------------------------------

\exercicio{Instrução \texttt{nop}}

Explique como a unidade de controle da CPU deve atuar para realizar a
instrução \texttt{nop}, detalhando os sinais de controle envolvidos.

\espacoresposta{3cm}

% --------------------------------------------------

\exercicio{Execução simultânea de instruções}

Analise se a CPU é capaz de executar, em um mesmo ciclo de clock,
as instruções \texttt{movw \%D, \%A} e \texttt{jg \%D}. Justifique sua
resposta com base na arquitetura da CPU, no caminho de dados
(\textit{datapath}) e nos sinais de controle envolvidos.

\espacoresposta{3cm}

% --------------------------------------------------

\exercicio{Sinal \texttt{loadPC}}

Explique quais sinais influenciam o carregamento do contador de programa
(\texttt{PC}) e como esse sinal afeta o fluxo de execução do programa.

\espacoresposta{3cm}

% ==================================================
\subsection{Execução e Debug da CPU}

\exercicio{Análise de falha em execução de programa}

Durante o desenvolvimento do projeto, o teste de integração
\texttt{./testeAssemblyMyCpu.py} falhou. A simulação executa o seguinte
programa:

\begin{verbatim}
leaw $0, %A
movw (%A), %D
leaw $1, %A
movw (%A), %A
addw %A, %D, (%A)
\end{verbatim}

A memória RAM inicial contém:

\begin{center}
\begin{tabular}{c c}
\toprule
Endereço & Dado \\
\midrule
0 & 0000000000000010 \\
1 & 0000000001000010 \\
3 & 0000000000000000 \\
\bottomrule
\end{tabular}
\end{center}

Com base na forma de onda dos sinais da CPU, identifique qual falha está
ocorrendo durante a execução do programa e explique o motivo.

\espacoresposta{3cm}

% --------------------------------------------------

\exercicio{Engenharia reversa de execução}

Você foi chamado para realizar engenharia reversa em um programa executando
no hardware do Z01. Na captura de dados, o sinal da instrução não foi obtido.

A partir da análise da forma de onda dos sinais da CPU, determine quais são
os \textbf{três comandos em assembly} que estão sendo executados no hardware
nesse momento.

\newpage
\newpage
\section{Linguagem de Máquina}
\setcounter{exercicio}{0}


\subsection{Conjunto de Instruções}

\exercicio{Classificação de comandos quanto ao suporte do hardware}

Classifique os comandos a seguir como \textbf{Suportado} ou
\textbf{Não suportado} pelo hardware Z01.1:

\begin{itemize}
  \item \texttt{leaw \$423, \%A}
  \item \texttt{leaw \$1, \%D}
  \item \texttt{movw \$-1, (\%A)}
  \item \texttt{leaw \$-15, \%A}
  \item \texttt{movw (\%A), \%A, \%D}
  \item \texttt{incw (\%A)}
  \item \texttt{addw (\%A), \%D, (\%A)}
  \item \texttt{jle \%A}
  \item \texttt{movw \$2, \%D}
  \item \texttt{movw \$1, \%D, \%A}
  \item \texttt{addw (\%A), \%D, \%D, \%A}
  \item \texttt{movw \%D, \%S}
\end{itemize}

\espacoresposta{3cm}

% --------------------------------------------------

\exercicio{Classificação de comandos em A ou C}

Classifique os comandos a seguir como instrução do tipo A ou do tipo C:

\begin{itemize}
  \item \texttt{leaw \$1, \%A}
  \item \texttt{movw \%D, \%A}
  \item \texttt{incw \%A}
  \item \texttt{jmp}
\end{itemize}

\newpage

\exercicio{Tradução da instrução \texttt{leaw}}

A instrução:
\[
\texttt{leaw \$5, \%A}
\]
é traduzida para qual palavra em binário?

\begin{itemize}
  \item \texttt{000000000000000000}
  \item \texttt{100000000000000101}
  \item \texttt{000000000000000101}
  \item \texttt{010101010101010101}
\end{itemize}

\espacoresposta{3cm}

% --------------------------------------------------

\exercicio{Tradução da instrução \texttt{movw}}

A instrução:
\[
\texttt{movw \%D, \%A}
\]
é traduzida para qual palavra em binário?

\begin{itemize}
  \item \texttt{1 000 1 001100 0 001 000}
  \item \texttt{1 000 0 001100 0 001 000}
  \item \texttt{0 000 1 001100 0 001 000}
  \item \texttt{1 000 0 000000 0 000 011}
\end{itemize}

\espacoresposta{3cm}

% --------------------------------------------------

\exercicio{Tradução da instrução \texttt{andw}}

A instrução:
\[
\texttt{andw \%A, \%D, \%D}
\]
é traduzida para qual palavra em binário?

\begin{itemize}
  \item \texttt{1 000 0 000000 0 010 000}
  \item \texttt{1 000 1 000000 0 010 000}
  \item \texttt{1 000 0 000000 0 001 000}
  \item \texttt{0 000 0 000000 0 010 001}
\end{itemize}

\espacoresposta{3cm}

% --------------------------------------------------

\exercicio{Tradução da instrução \texttt{andw} com dois destinos}

A instrução:
\[
\texttt{andw \%A, \%D, \%D, \%A}
\]
é traduzida para qual palavra em binário?

\begin{itemize}
  \item \texttt{1 000 0 000000 0 010 000}
  \item \texttt{1 000 0 000000 0 011 000}
  \item \texttt{1 000 0 000000 0 001 000}
  \item \texttt{0 000 0 000000 0 010 001}
\end{itemize}

\espacoresposta{3cm}

% --------------------------------------------------

\exercicio{Tradução da instrução \texttt{jmp}}

A instrução \texttt{jmp} é traduzida para qual palavra em binário?

\begin{itemize}
  \item \texttt{1 000 0 XXXXXX 0 000 111}
  \item \texttt{1 000 0 XXXXXX 0 001 000}
  \item \texttt{0 000 0 XXXXXX 0 000 111}
  \item \texttt{1 000 0 XXXXXX 0 000 000}
\end{itemize}

\espacoresposta{3cm}

% --------------------------------------------------

\exercicio{Interpretação de código em linguagem de máquina}

O código a seguir está em linguagem de máquina do processador Z01. Considere que, neste hardware:
\begin{itemize}
  \item o endereço \texttt{RAM[16384]} corresponde às chaves de entrada (\texttt{SW});
  \item o endereço \texttt{RAM[16385]} corresponde aos LEDs de saída.
\end{itemize}

O que o código abaixo faz?

\begin{center}
\ttfamily
\begin{tabular}{l}
000101001011000001 \\
100011100000010000 \\
000101001011000000 \\
100000011000100000 \\
000000000000000000 \\
100000011000000111 \\
100001010100000000
\end{tabular}
\end{center}

\begin{enumerate}
  \item Move o valor da \texttt{RAM[3]} para \texttt{RAM[2]} deixando-o negativo
  \item Escreve um pixel no LCD
  \item Copia o valor das chaves \texttt{SW} para os LEDs
  \item Adiciona o valor de \texttt{RAM[1]} com \texttt{RAM[2]}
\end{enumerate}

\textit{Comentário:} este exercício explora o conceito de periféricos
mapeados em memória, em que dispositivos de entrada e saída compartilham
o mesmo espaço de endereçamento da RAM.


\espacoresposta{2cm}



\newpage

% ====================
% Gabarito
% ====================
\section*{Gabarito Comentado}
\addcontentsline{toc}{section}{Gabarito Comentado}

\addcontentsline{toc}{subsection}{Álgebra Booleana}
\section*{Álgebra Booleana}
\setcounter{exercicio}{0}

\gabaritoex{1}
Portas lógicas correspondentes: AND, NAND, OR, NOR, NOT, XOR e XNOT.

\gabaritoex{2}
(a) A tabela representa a porta XOR, pois a saída é 1 quando exatamente uma das
entradas é 1.  
(b) A tabela representa a porta NAND, pois apenas quando ambas as entradas são 1
a saída é 0.

\gabaritoex{3}
Aplicando a distributividade:
\[
A(B+\overline{C}+D)=AB+A\overline{C}+AD
\]

\gabaritoex{4}
Pelo teorema de De Morgan:
\[
\overline{A+B+C}=\overline{A}\,\overline{B}\,\overline{C}
\]

\gabaritoex{5}
As simplificações incorretas são aquelas que violam a distributividade ou a
absorção, em especial quando fatores são omitidos durante a expansão. Nesse caso, as alternativas 1 e 3

\gabaritoex{6}
A propriedade falsa é a alternativa que não pode ser deduzida a partir das leis
fundamentais da álgebra booleana. A alternativa 2

\gabaritoex{7}
A forma canônica correta é:
\[
Q=\overline{A}\,\overline{B}+AB
\]

\gabaritoex{8}
A expressão simplificada corresponde a:
\space

PoS:
\[
Q = (A + B + \overline{C})
    (A + \overline{B} + C)
    (\overline{A} + B + \overline{C})
    (\overline{A} + \overline{B} + C)
\]

SoP:
\[
Q = \overline{A}\,\overline{B}\,\overline{C}
  + \overline{A} B C
  + A \overline{B}\,\overline{C}
  + A B C
\]
\[Q = 
BC((\overline{A} + A) + \overline{BC}(\overline{A} + A)\]
\[Q = BC + \overline{BC}\]

\begin{figure}
    \centering
    \includegraphics[width=0.5\linewidth]{fotos/circuito1.png}
    \label{fig:placeholder}
\end{figure}

\gabaritoex{9}
4 combinações já que duas são iguais

\gabaritoex{10}
A principal característica da forma canônica SoP é permitir a implementação do
circuito utilizando apenas portas AND, OR e inversores.

\gabaritoex{11}
\[1.\overline{BC} + B\overline{C} + \overline{A}BC\]
\[2.\overline{C}(\overline{B}+B) + \overline{A}BC\]
\[3.\overline{C} + \overline{A}BC\]

\gabaritoex{12}
Mapa 1: \[B+A\]
Mapa 2: \[A\overline{B} + \overline{B}C + \overline{A}B\]
Mapa 3: \[\overline{BCD} +A\overline{BC} + BC\overline{D}\]
Mapa 4: \[A + C\overline{D} + \overline{BD}\]

\gabaritoex{13}

\begin{figure}[H]
    \centering
    \includegraphics[width=0.5\linewidth]{fotos/mapa2.png}
\end{figure}

\[
Q = \overline{A} + \overline{B} + \overline{C} + \overline{D}
\]


\gabaritoex{14}
\[
\begin{array}{c c c|c}
A & B & C & Q \\ \hline
0 & 0 & 0 & 0 \\
0 & 0 & 1 & 0 \\
0 & 1 & 0 & 0 \\
0 & 1 & 1 & 1 \\
1 & 0 & 0 & 0 \\
1 & 0 & 1 & 1 \\
1 & 1 & 0 & 0 \\
1 & 1 & 1 & 1
\end{array}
\]

\[
Q = \overline{A}BC + A\overline{B}C + ABC
\]

\[
Q = BC(\overline{A} + A) + A\overline{B}C
\]
\[
Q = BC + A\overline{B}C
\]


\newpage

\addcontentsline{toc}{subsection}{Dados Digitais}
\section*{Dados Digitais}
\setcounter{exercicio}{0}

\gabaritoex{1}
O endereço \texttt{0x0C} \[(12)\] contém o valor binário:
\[
0b11111011
\]
que corresponde ao valor decimal 251.

\gabaritoex{2}
Os bytes capturados são:
\[
0x5A,\;0x69,\;0x67,\;0x61
\]
que correspondem aos caracteres ASCII \texttt{Ziga}.

\gabaritoex{3}
Os valores exibidos no display de sete segmentos são:
\[
1,\;0,\;2
\]

\gabaritoex{4}
\begin{itemize}
  \item nibble: 4 bits
  \item byte: 8 bits
  \item halfword: 16 bits
  \item word: 32 bits
\end{itemize}

\gabaritoex{5}
Conversões corretas para binário, com número mínimo de bits, conforme tabela
padrão de potências de dois

\begin{center}
\begin{tabular}{|c|c|c|}
\hline
\textbf{Decimal} & \textbf{nº de bits} & \textbf{Binário} \\ \hline
0     & 1  & $0b0$ \\
1     & 1  & $0b1$ \\
2     & 2  & $0b10$ \\
3     & 2  & $0b11$ \\
4     & 3  & $0b100$ \\
5     & 3  & $0b101$ \\
6     & 3  & $0b110$ \\
7     & 3  & $0b111$ \\
8     & 4  & $0b1000$ \\
9     & 4  & $0b1001$ \\
10    & 4  & $0b1010$ \\
115   & 7  & $0b1110011$ \\
256   & 9  & $0b100000000$ \\
1027  & 11 & $0b10000000011$ \\ \hline
\end{tabular}
\end{center}
.

\gabaritoex{6}
\[
0b0=0,\quad 0b100=4,\quad 0b10011=19,\quad 0b11111=31,\quad 0b01010=10
\]

\gabaritoex{7}
\begin{center}
\begin{tabular}{|c|c|}
\hline
\textbf{Decimal} & \textbf{Hexadecimal} \\ \hline
0  & $0x0$  \\
1  & $0x1$  \\
2  & $0x2$  \\
3  & $0x3$  \\
4  & $0x4$  \\
5  & $0x5$  \\
6  & $0x06$ \\
7  & $0x7$  \\
8  & $0x8$  \\
9  & $0x9$  \\
10 & $0xA$  \\
11 & $0xB$  \\
12 & $0xC$  \\
13 & $0xD$  \\
14 & $0xE$  \\
15 & $0xF$  \\
16 & $0x10$ \\
17 & $0x11$ \\
18 & $0x12$ \\ \hline
\end{tabular}
\end{center}


\gabaritoex{8}
\[
0b11110=0x1E,\quad 0b101=0x5,\quad 0b10100011=0xA3,\quad 0b11010=0x1A, \quad 0b00000=0x0
\]

\gabaritoex{9}
\[
0x0003=3,\quad 0xA=10,\quad 0x55=85,\quad 0x0101=257
\]

\gabaritoex{10}

\begin{center}
\begin{tabular}{|c|c|}
\hline
\textbf{Decimal} & \textbf{Octal} \\ \hline
0  & $0o0$  \\
1  & $0o1$  \\
2  & $0o2$  \\
3  & $0o3$  \\
4  & $0o4$  \\
5  & $0o5$  \\
6  & $0o06$ \\
7  & $0o7$  \\
8  & $0o10$ \\
9  & $0o11$ \\
10 & $0o12$ \\
11 & $0o13$ \\
12 & $0o14$ \\
13 & $0o15$ \\
14 & $0o16$ \\
15 & $0o17$ \\
16 & $0o20$ \\
17 & $0o21$ \\
18 & $0o22$ \\ \hline
\end{tabular}
\end{center}


\gabaritoex{11}

\begin{center}
\begin{tabular}{|c|c|}
\hline
\textbf{Decimal} & \textbf{BCD} \\ \hline
50  & $0101\ 0000$ \\
1   & $0001$ \\
103 & $0001\ 0000\ 0011$ \\
904 & $1001\ 0000\ 0100$ \\
4   & $0100$ \\
5   & $0101$ \\ \hline
\end{tabular}
\end{center}


\gabaritoex{12}
\begin{center}
\begin{tabular}{c|c}
\hline
\textbf{Binário} & \textbf{Decimal} \\
\hline
0b100 & 4 \\
0b0 & 0 \\
0b10011 & 19 \\
0b11111 & 31 \\
0b01010 & 10
\end{tabular}
\end{center}

\[
`1'=00110001,\;`a'=01100001,\;`A'=01000001,\;`Z'=01011010,\;`0'=00110000
\]
\newpage

\addcontentsline{toc}{subsection}{Sistemas Síncronos}
\section*{Sistemas Síncronos e Elementos de Memória}
\setcounter{exercicio}{0}

\gabaritoex{1}
Circuitos síncronos utilizam um sinal de clock para sincronizar mudanças de
estado, enquanto circuitos assíncronos respondem imediatamente às variações
de entrada.

\gabaritoex{2}
O clock define o ritmo de operação do sistema e garante sincronização entre os
elementos sequenciais.

\gabaritoex{3}
Latches são sensíveis ao nível do sinal de controle, enquanto flip-flops mudam
de estado apenas na borda do clock.

\gabaritoex{4}
O flip-flop tipo D armazena o valor da entrada D na borda ativa do clock.

\gabaritoex{5}
A tabela verdade do FF-D mostra que a saída assume o valor de D apenas na
transição do clock.

\gabaritoex{6}
Q(A) = 0,\quad
Q(B) = 1,\quad
Q(C) = 0,\quad
Q(D) = 0,\quad
Q(E) = 1


\gabaritoex{7}
Um registrador é um conjunto de bits capaz de armazenar um valor binário.

\gabaritoex{8}
O sinal de endereço indica qual posição da memória será acessada.

\gabaritoex{9}

\begin{center}
\begin{tabular}{|c|c|c|}
\hline
\textbf{Clock} & \textbf{Entrada} & \textbf{Bit armazenado} \\ \hline
$\uparrow$ & 0 & 0 \\ 
$\uparrow$ & 1 & 1 \\ 
0/1        & X & mantém valor \\ \hline
\end{tabular}
\end{center}

\textit{O Binary Digit armazena o valor da entrada apenas na borda de subida
do clock, mantendo o valor anterior nos demais instantes.}


\newpage

\addcontentsline{toc}{subsection}{Aritmética Binária e Somadores}
\section*{Aritmética Binária e Somadores}
\setcounter{exercicio}{0}

\gabaritoex{1}
\[
-5=11111011
\]
O valor $-230$ não pode ser representado em 8 bits pois excede o intervalo
$[-128,127]$.  
O valor $10000011$ representa um número negativo.

\gabaritoex{2}
A letra \texttt{G} corresponde ao valor ASCII:
\[
01000111
\]

\gabaritoex{3}
\[
10010111+01100010=11111001=-7
\]

\gabaritoex{4}
Ocorre overflow, pois dois números positivos geraram um resultado negativo.

\gabaritoex{5}

A partir da tabela verdade, observa-se que:

\begin{itemize}
  \item a saída \textbf{soma} é igual a 1 quando exatamente uma das entradas
  é igual a 1, caracterizando a operação \textbf{XOR};
  \item a saída \textbf{carry} é igual a 1 somente quando ambas as entradas
  são iguais a 1, caracterizando a operação \textbf{AND}.
\end{itemize}

As expressões booleanas do Half-Adder são:
\[
\text{soma} = a \oplus b
\qquad
\text{carry} = a \cdot b
\]

\textbf{Circuito lógico do Half-Adder:}

\begin{center}
\begin{tikzpicture}[circuit logic US, scale=1, every circuit symbol/.style={scale=1}]

% Entradas
\node (a) at (0,1) {$a$};
\node (b) at (0,0) {$b$};

% Portas
\node[xor gate, draw] (xor1) at (2,0.75) {};
\node[and gate, draw] (and1) at (2,-0.75) {};

% Saídas
\node (soma)  at (4,0.75) {$\text{soma}$};
\node (carry) at (4,-0.75) {$\text{carry}$};

% Conexões
\draw (a) -- (xor1.input 1);
\draw (b) -- (xor1.input 2);
\draw (xor1.output) -- (soma);

\draw (a) |- (and1.input 1);
\draw (b) |- (and1.input 2);
\draw (and1.output) -- (carry);

\end{tikzpicture}
\end{center}


\textit{O Half-Adder é o bloco básico dos somadores binários e serve como
base para a construção do Full-Adder e de somadores multi-bit.}


\gabaritoex{6}

A partir da tabela verdade, o Full-Adder soma três bits de entrada
($a$, $b$ e $c_{in}$), produzindo as saídas \textbf{soma} e \textbf{carry}.

As expressões booleanas são:

\[
\text{soma} = a \oplus b \oplus c_{in}
\]

\[
\text{carry} = (a \cdot b) + \bigl(c_{in} \cdot (a \oplus b)\bigr)
\]

\textbf{Circuito lógico do Full-Adder:}

\begin{center}
\begin{circuitikz}[scale=1]

% Entradas
\node (a) at (0,2) {$a$};
\node (b) at (0,1) {$b$};
\node (cin) at (0,0) {$c_{in}$};

% XORs
\draw (1.5,1.5) node[xor port] (xor1) {};
\draw (3.5,1.5) node[xor port] (xor2) {};

% ANDs
\draw (1.5,0.5) node[and port] (and1) {};
\draw (3.5,0.5) node[and port] (and2) {};

% OR
\draw (5.5,0.5) node[or port] (or1) {};

% Conexões soma
\draw (a) -- (xor1.in 1);
\draw (b) -- (xor1.in 2);
\draw (xor1.out) -- (xor2.in 1);
\draw (cin) |- (xor2.in 2);
\draw (xor2.out) -- ++(1,0) node[right]{soma};

% Conexões carry
\draw (a) |- (and1.in 1);
\draw (b) |- (and1.in 2);
\draw (cin) |- (and2.in 2);
\draw (xor1.out) |- (and2.in 1);
\draw (and1.out) -- (or1.in 1);
\draw (and2.out) -- (or1.in 2);
\draw (or1.out) -- ++(1,0) node[right]{carry};

\end{circuitikz}
\end{center}



\textit{O Full-Adder pode ser visto como a combinação de dois Half-Adders
e uma porta OR para o cálculo do carry.}


\gabaritoex{7}

O somador de 2 bits é construído utilizando dois Full-Adders conectados
em cascata, permitindo a propagação do sinal de carry.

\textbf{Circuito lógico do Somador de 2 bits:}

\begin{center}
\begin{tikzpicture}[scale=1, every node/.style={font=\small}]

% Full-Adder do bit menos significativo
\node[draw, minimum width=2.4cm, minimum height=1.2cm] (FA0) at (0,2) {FA};

\node[left] at ($(FA0.west)+(0,0.4)$) {$X_0$};
\node[left] at ($(FA0.west)+(0,0)$) {$Y_0$};
\node[left] at ($(FA0.west)+(0,-0.4)$) {$c_0 = 0$};

\node[right] at (FA0.east) {$S_0$};

% Full-Adder do bit mais significativo
\node[draw, minimum width=2.4cm, minimum height=1.2cm] (FA1) at (0,0) {FA};

\node[left] at ($(FA1.west)+(0,0.3)$) {$X_1$};
\node[left] at ($(FA1.west)+(0,-0.1)$) {$Y_1$};

\node[right] at ($(FA1.east)+(0,0.3)$) {$S_1$};
\node[right] at ($(FA1.east)+(0,-0.3)$) {$c_{out}$};

% Carry entre os estágios
\draw[->, thick] (FA0.south) -- node[right] {$c_1$} (FA1.north);

\end{tikzpicture}
\end{center}


\textit{Esse tipo de somador é conhecido como \textbf{Ripple Carry Adder},
pois o carry gerado em cada estágio se propaga para o próximo.}


\newpage

\addcontentsline{toc}{subsection}{Arquitetura e CPU}
\section*{Arquitetura, Execução e Debug da CPU}
\setcounter{exercicio}{0}

\gabaritoex{1}
Para atender aos requisitos propostos, a arquitetura original da CPU do Z01.1
deve ser estendida conforme descrito a seguir:

Um novo registrador \texttt{\%S} deve ser inserido no banco de registradores da CPU,
de forma análoga aos registradores existentes \texttt{\%A} e \texttt{\%D}.
Esse registrador deve possuir largura de 16 bits e estar conectado ao barramento
interno de dados.

A linguagem de máquina deve ser estendida para permitir que o registrador
\texttt{\%S} seja selecionado como origem ou destino em instruções do tipo C.
Isso implica a ampliação dos campos de seleção de registradores na instrução
(binários adicionais no campo de destino e/ou origem).

Para permitir a instrução \texttt{movw \%A, (\%D)}, o registrador \texttt{\%D}
deve ser conectado ao barramento de endereços da memória RAM, funcionando como
um registrador de endereçamento alternativo ao \texttt{\%A}.

Para permitir a instrução \texttt{leaw \$5, \%D}, a lógica de carregamento imediato
deve ser estendida de modo que o valor proveniente da ROM possa ser direcionado
não apenas ao registrador \texttt{\%A}, mas também ao registrador \texttt{\%D}.
\begin{center}
\begin{tikzpicture}[scale=0.9]

% Registradores
\node[draw, minimum width=1.6cm] (A) at (0,2) {\%A};
\node[draw, minimum width=1.6cm] (D) at (0,1) {\%D};
\node[draw, minimum width=1.6cm] (S) at (0,0) {\%S};

% Barramento
\draw[thick] (2,-0.5) -- (2,2.5);

% CPU
\node[draw, minimum width=3cm, minimum height=4cm] (CPU) at (5,1) {};
\node at (5,2.4) {Control Unit};
\node[draw, minimum width=2cm] (ALU) at (5,1) {ULA};

% Memórias
\node[draw] (ROM) at (8,2.2) {ROM};
\node[draw] (RAM) at (8,-0.2) {RAM};

% Conexões
\draw (A.east) -- (2,2);
\draw (D.east) -- (2,1);
\draw (S.east) -- (2,0);

\draw (2,1) -- (ALU.west);
\draw (ALU.east) -- (RAM.west);
\draw (CPU.north east) -- (ROM.west);

% Endereçamento via D
\draw[dashed] (D.south) |- node[below]{addr} (RAM.south);

\end{tikzpicture}
\end{center}


\textit{A linha tracejada indica o novo caminho de endereçamento da RAM via o registrador \%D.}


\gabaritoex{2}
A instrução \texttt{nop} não altera o estado do sistema, apenas avança o PC.

\gabaritoex{3}
A CPU não executa simultaneamente essas instruções devido à arquitetura
sequencial.

\gabaritoex{4}
O sinal \texttt{loadPC} controla o avanço ou salto do contador de programa.

\gabaritoex{5}
A falha ocorre devido à ativação incorreta do sinal de carregamento do
registrador \%A durante a execução da instrução \texttt{addw \%A, \%D, (\%A)}.
O registrador \%A deveria apenas fornecer o endereço de memória, mas acaba
sendo sobrescrito no mesmo ciclo, resultando em acesso inválido à RAM.


\gabaritoex{6}
Os comandos executados são:
\begin{verbatim}
leaw $1, %A
movw (%A), %D
addw %D, %A, %A
\end{verbatim}
\newpage

\addcontentsline{toc}{subsection}{Linguagem de Máquina}
\section*{Linguagem de Máquina}
\setcounter{exercicio}{0}

\gabaritoex{1}
Apenas instruções compatíveis com a ISA do Z01 são suportadas.

\gabaritoex{2}
\begin{itemize}
  \item leaw \$1, \%A — instrução A
  \item movw \%D, \%A — instrução C
  \item incw \%A — instrução C
  \item jmp — instrução C
\end{itemize}

\gabaritoex{3}
A instrução \texttt{leaw \$5, \%A} é traduzida como:
\[
000000000000000101
\]

\gabaritoex{4}
A tradução correta é:
\[
1\;000\;0\;001100\;0\;001\;000
\]

\gabaritoex{5}
A tradução correta é:
\[
1\;000\;0\;000000\;0\;010\;000
\]

\gabaritoex{6}
A tradução correta é:
\[
1\;000\;0\;000000\;0\;011\;000
\]

\gabaritoex{7}
A instrução \texttt{jmp} corresponde a:
\[
1\;000\;0\;010101\;0\;000\;111
\]

\gabaritoex{8}
A instrução \texttt{jl} corresponde a:
\[
1\;000\;0\;001100\;0\;000\;100
\]

\gabaritoex{9}
O código copia o valor das chaves \texttt{SW} para os LEDs.

\newpage

% ====================
% Materiais complementares
% ====================
\section*{Para continuar estudando}
\addcontentsline{toc}{section}{Para continuar estudando}

Os materiais abaixo foram indicados ao longo da disciplina ou utilizados como
apoio para aprofundamento dos conceitos apresentados neste caderno.

\subsection*{Livros}
\begin{itemize}
  \item NISAN, N.; SCHOCKEN, S. \textit{The Elements of Computing Systems}.
  \item TOCCI, R.; WIDMER, N.; MOSS, G. \textit{Digital Systems}.
  \item FLOYD, T. \textit{Digital Fundamentals}.
\end{itemize}

\subsection*{Vídeos}
\begin{itemize}
  \item Logic 101 — Truth Tables (YouTube)
  \item Computer Science: Karnaugh Maps — Introduction
  \item Coursera — Build a Computer from First Principles
\end{itemize}

\subsection*{Leitura complementar}
\begin{itemize}
  \item Don Lancaster — \textit{RTL Cookbook}
  \item Artigos e materiais sobre a arquitetura Z01
\end{itemize}

\end{document}
